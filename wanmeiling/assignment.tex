\documentclass[a4,oneside]{article}
\usepackage[winfonts,cap]{ctex}

\setCJKmainfont[BoldFont={SimHei},ItalicFont={KaiTi}]{SimSun}
\setCJKsansfont{SimHei}
\setCJKmonofont{FangSong}

\setCJKfamilyfont{zhsong}{SimSun}
\setCJKfamilyfont{zhhei}{SimHei}
\setCJKfamilyfont{zhkai}{KaiTi}
\setCJKfamilyfont{zhfs}{FangSong}
\pagestyle{empty}
\usepackage{eepic}
\usepackage{graphicx}
\usepackage[abs]{overpic}

%设置页面。
%\setlength{\paperwidth}{210mm}
%\setlength{\paperheight}{297mm}

%设置正文大小
%\setlength{\textwidth}{208mm}
%\setlength{\textheight}{295mm}
\setlength{\textwidth}{210mm}
\setlength{\textheight}{297mm}

%先设置正文区在正中间
\setlength{\oddsidemargin}{-0.8in}
\setlength{\evensidemargin}{-0.8in}

\setlength{\topmargin}{-1.2in}
%\setlength{\hoffset}{0mm}

\begin{document}
\begin{center}
\setlength{\unitlength}{0.1mm}


\Large
\begin{overpic}{eps/assignment_blank_1.eps}
% \put(0,0){\circle*{20}}
\put(330,1892){\makebox(320,0)[c]{康尼}}
\put(1150,1892){\makebox(0,0){通信工程(计算机通信)}}
% \put(704,1805){\framebox{\parbox[t][25mm][t]{85mm}{\setlength{\baselineskip}{10mm} \center 网上商城购物子系统的设计与实现}}}
\put(704,1765){%
\begin{minipage}[t][35mm][t]{85mm}
\setlength{\baselineskip}{10mm}
\centering 网上商城购物子系统的设计与实现
\end{minipage}
}
\put(704,1715){\makebox(848,200){}}
\put(1087,1453){\makebox(0,0){龚泉枫}}
\put(1087,1339){\makebox(0,0){K算通081}}
\put(1087,1232){\makebox(0,0){2012.2.20--2012.6.8}}
%\put(1087,1120){\makebox(0,0){专业实验室}}
%\put(1087,1008){\makebox(0,0){王少东}}
\put(1087,1120){\makebox(0,0){王少东}}
\put(1029,571){\makebox(0,0){2012}}
\put(1227,571){\makebox(0,0){02}}
\put(1348,571){\makebox(0,0){18}}
\end{overpic}
\begin{overpic}{eps/assignment_blank_2.eps}
% \put(0,0){\circle*{20}}
\put(297,2157){\parbox[b][27mm][t]{162mm}{
\setlength{\baselineskip}{9mm} 

}}
\put(297,1441){\parbox[b][56mm][t]{130mm}{
\setlength{\baselineskip}{9mm} 
\CTEXindent

}}
\put(297,922){\parbox[b][45mm][t]{162mm}{
\setlength{\baselineskip}{9mm} 

}}
\put(297,400){\parbox[b][44mm][t]{142mm}{
\setlength{\baselineskip}{9mm} 

}}
\end{overpic}
% \normalsize
\large
\begin{overpic}{eps/assignment_blank_3.eps}
% \put(0,0){\circle*{20}}
\put(315,737){\parbox[b][157mm][t]{139mm}{
\renewcommand{\arraystretch}{1.3}
\begin{tabular}{p{28mm}p{80mm}p{30mm}}
	&
        	&
 \\	&
        	&
 \\	&
        	&
 \\	&
        	&
 \\	&
        	&
 \\	&
        	&
 \\	&
        	&
 \\	&
        	&
 \\	&
        	&
 \\	&
        	&
 \\
\end{tabular}
}}
\end{overpic}
\Large
\begin{overpic}{eps/assignment_blank_1.eps}
% \put(0,0){\circle*{20}}
\put(330,1892){\makebox(320,0)[c]{康尼}}
\put(1150,1892){\makebox(0,0){通信工程(计算机通信)}}
% \put(704,1805){\framebox{\parbox[t][25mm][t]{85mm}{\setlength{\baselineskip}{10mm} \center 一种应用域描述语言的构造与解释}}}
\put(704,1765){%
\begin{minipage}[t][35mm][t]{85mm}
\setlength{\baselineskip}{10mm}
\centering 一种应用域描述语言的构造与解释
\end{minipage}
}
\put(704,1715){\makebox(848,200){}}
\put(1087,1453){\makebox(0,0){华杰}}
\put(1087,1339){\makebox(0,0){K算通081}}
\put(1087,1232){\makebox(0,0){2012.2.20--2012.6.8}}
%\put(1087,1120){\makebox(0,0){专业实验室}}
%\put(1087,1008){\makebox(0,0){王少东}}
\put(1087,1120){\makebox(0,0){王少东}}
\put(1029,571){\makebox(0,0){2012}}
\put(1227,571){\makebox(0,0){02}}
\put(1348,571){\makebox(0,0){18}}
\end{overpic}
\begin{overpic}{eps/assignment_blank_2.eps}
% \put(0,0){\circle*{20}}
\put(297,2157){\parbox[b][27mm][t]{162mm}{
\setlength{\baselineskip}{9mm} 

试卷排版项目案例\par
特定应用领域描述语言需求
}}
\put(297,1441){\parbox[b][56mm][t]{130mm}{
\setlength{\baselineskip}{9mm} 
\CTEXindent

通用程序设计预研难以满足特定应用场合的需求。希望设计一种面向特定问题的语言
DSL(Domain Specific Language),来处理特定任务。\par
要求以考试试卷这类具有特殊格式的教学文档的设计为例,
设计一种面向特定应用语言,设计并实现一个该语言的解释器。
}}
\put(297,922){\parbox[b][45mm][t]{162mm}{
\setlength{\baselineskip}{9mm} 

毕业设计开题报告,3000字以上。\par
外文参考资料的译文,2000字以上。\par
毕业设计说明书,15000字以上。\par
设计成果及其数字化存储。
}}
\put(297,400){\parbox[b][44mm][t]{142mm}{
\setlength{\baselineskip}{9mm} 

[1]  Eric S. Raymond. Making Pictures With GNU PIC[EB/OL]. http://floppsie.comp.glam.ac.uk/Glamorgan/gaius/web/pic.html \par
[2]  胡圣明, 李青山, 陈平. 基于对象的特定域语言构造方法[J]. 西安电子科技大学学报, 2006(1). \par
[3]  周艳明. 基于领域专用语言的应用软件自动生成[J]. 计算机工程与应用, 2003(10).
}}
\end{overpic}
% \normalsize
\large
\begin{overpic}{eps/assignment_blank_3.eps}
% \put(0,0){\circle*{20}}
\put(315,737){\parbox[b][157mm][t]{139mm}{
\renewcommand{\arraystretch}{1.3}
\begin{tabular}{p{28mm}p{80mm}p{30mm}}

	第一周至第三周	&
	根据任务书要求,搜集、查阅相关资料,完成开题报告;完成外文文献翻译工作
	&
 \\
	第四周至第五周	&
	深入研究文献,准备工作环境\par
描述DSL的需求\par
研究不同的实现方案和路径,确定实现方式
	&
 \\
	第六周至第七周	&
	研究手工方式生成目标文档的各种步骤,发现其中的局限\par
设计一个微型的语言系统(词法、语法),尝试它的解释器设计\par
使用DSL设计目标文档
	&
 \\
	第八周至第九周	&
	丰富词法和语法,演变成完整的系统\par
建立测试用例\par
设计测试方法和测试过程
	&
 \\
	第十周至第十二周	&
	完善解释器程序,优化DSL\par
健壮性改进与测试\par
形成测试报告
	&
 \\
	第十三周至第十四周	&
	撰写毕业设计说明书\par
做好答辩准备
	&
 \\
	第十五周	&
	毕业设计说明书修订与完善\par
论文评阅
	&
 \\
	第十六周	&
	论文答辩\par
毕业设计材料归档
	&
 \\	&
        	&
 \\	&
        	&
 \\
\end{tabular}
}}
\end{overpic}
\Large
\begin{overpic}{eps/assignment_blank_1.eps}
% \put(0,0){\circle*{20}}
\put(330,1892){\makebox(320,0)[c]{康尼}}
\put(1150,1892){\makebox(0,0){通信工程(计算机通信)}}
% \put(704,1805){\framebox{\parbox[t][25mm][t]{85mm}{\setlength{\baselineskip}{10mm} \center 电话银行用户签约管理系统的设计与实现}}}
\put(704,1765){%
\begin{minipage}[t][35mm][t]{85mm}
\setlength{\baselineskip}{10mm}
\centering 电话银行用户签约管理系统的设计与实现
\end{minipage}
}
\put(704,1715){\makebox(848,200){}}
\put(1087,1453){\makebox(0,0){强刚}}
\put(1087,1339){\makebox(0,0){K算通081}}
\put(1087,1232){\makebox(0,0){2012.2.20--2012.6.8}}
%\put(1087,1120){\makebox(0,0){专业实验室}}
%\put(1087,1008){\makebox(0,0){王少东}}
\put(1087,1120){\makebox(0,0){王少东}}
\put(1029,571){\makebox(0,0){2012}}
\put(1227,571){\makebox(0,0){02}}
\put(1348,571){\makebox(0,0){18}}
\end{overpic}
\begin{overpic}{eps/assignment_blank_2.eps}
% \put(0,0){\circle*{20}}
\put(297,2157){\parbox[b][27mm][t]{162mm}{
\setlength{\baselineskip}{9mm} 

}}
\put(297,1441){\parbox[b][56mm][t]{130mm}{
\setlength{\baselineskip}{9mm} 
\CTEXindent

}}
\put(297,922){\parbox[b][45mm][t]{162mm}{
\setlength{\baselineskip}{9mm} 

}}
\put(297,400){\parbox[b][44mm][t]{142mm}{
\setlength{\baselineskip}{9mm} 

}}
\end{overpic}
% \normalsize
\large
\begin{overpic}{eps/assignment_blank_3.eps}
% \put(0,0){\circle*{20}}
\put(315,737){\parbox[b][157mm][t]{139mm}{
\renewcommand{\arraystretch}{1.3}
\begin{tabular}{p{28mm}p{80mm}p{30mm}}
	&
        	&
 \\	&
        	&
 \\	&
        	&
 \\	&
        	&
 \\	&
        	&
 \\	&
        	&
 \\	&
        	&
 \\	&
        	&
 \\	&
        	&
 \\	&
        	&
 \\
\end{tabular}
}}
\end{overpic}
\Large
\begin{overpic}{eps/assignment_blank_1.eps}
% \put(0,0){\circle*{20}}
\put(330,1892){\makebox(320,0)[c]{通信工程}}
\put(1150,1892){\makebox(0,0){通信工程(多媒体通信)}}
% \put(704,1805){\framebox{\parbox[t][25mm][t]{85mm}{\setlength{\baselineskip}{10mm} \center 轻量级标记语言在持久文档中的应用研究}}}
\put(704,1765){%
\begin{minipage}[t][35mm][t]{85mm}
\setlength{\baselineskip}{10mm}
\centering 轻量级标记语言在持久文档中的应用研究
\end{minipage}
}
\put(704,1715){\makebox(848,200){}}
\put(1087,1453){\makebox(0,0){万美玲}}
\put(1087,1339){\makebox(0,0){媒体通信081}}
\put(1087,1232){\makebox(0,0){2012.2.20--2012.6.8}}
%\put(1087,1120){\makebox(0,0){专业实验室}}
%\put(1087,1008){\makebox(0,0){王少东}}
\put(1087,1120){\makebox(0,0){王少东}}
\put(1029,571){\makebox(0,0){2012}}
\put(1227,571){\makebox(0,0){02}}
\put(1348,571){\makebox(0,0){18}}
\end{overpic}
\begin{overpic}{eps/assignment_blank_2.eps}
% \put(0,0){\circle*{20}}
\put(297,2157){\parbox[b][27mm][t]{162mm}{
\setlength{\baselineskip}{9mm} 

一种持久文档的案例\par
特定应用领域描述语言需求
}}
\put(297,1441){\parbox[b][56mm][t]{130mm}{
\setlength{\baselineskip}{9mm} 
\CTEXindent

比较现有的各种标记语言,选则一种人类可读的轻量级标记语言,
考察其在给定场合下的应用可能与限制。重点考察Markdown标记语言的应用。\par
要求以常见教学文档的为例,
设计一种面向特定应用文档描述规范,设计并实现一个
向通用格式的转换程序。
}}
\put(297,922){\parbox[b][45mm][t]{162mm}{
\setlength{\baselineskip}{9mm} 

毕业设计开题报告,3000字以上。\par
外文参考资料的译文,2000字以上。\par
毕业设计说明书,15000字以上。\par
设计成果及其数字化存储。
}}
\put(297,400){\parbox[b][44mm][t]{142mm}{
\setlength{\baselineskip}{9mm} 

[1]  John Gruber. Markdown home page[EB/OL]. http://daringfireball.net/projects/markdown/\par
[2]  胡圣明, 李青山, 陈平. 基于对象的特定域语言构造方法[J]. 西安电子科技大学学报, 2006(1). \par
[3]  周艳明. 基于领域专用语言的应用软件自动生成[J]. 计算机工程与应用, 2003(10).
}}
\end{overpic}
% \normalsize
\large
\begin{overpic}{eps/assignment_blank_3.eps}
% \put(0,0){\circle*{20}}
\put(315,737){\parbox[b][157mm][t]{139mm}{
\renewcommand{\arraystretch}{1.3}
\begin{tabular}{p{28mm}p{80mm}p{30mm}}

	第一周至第三周	&
	根据任务书要求,搜集、查阅相关资料,完成开题报告;完成外文文献翻译工作
	&
 \\
	第四周至第五周	&
	深入研究文献,准备工作环境\par
书面描述目标需求\par
研究几种不同的标记语言,确定实现方式
	&
 \\
	第六周至第七周	&
	研究轻量级标记语言标记目标文档的方法\par
设计一套基本的标记规范,尝试它的转换器设计\par
使用自己设计的标记规范设计目标文档
	&
 \\
	第八周至第九周	&
	丰富完善标记系统,演变成完整的系统\par
建立测试用例\par
设计测试方法和测试过程
	&
 \\
	第十周至第十二周	&
	完善转换器程序,优化标记规范\par
健壮性改进与测试\par
形成测试报告
	&
 \\
	第十三周至第十四周	&
	撰写毕业设计说明书\par
做好答辩准备
	&
 \\
	第十五周	&
	毕业设计说明书修订与完善\par
论文评阅
	&
 \\
	第十六周	&
	论文答辩\par
毕业设计材料归档
	&
 \\	&
        	&
 \\	&
        	&
 \\
\end{tabular}
}}
\end{overpic}
\Large
\begin{overpic}{eps/assignment_blank_1.eps}
% \put(0,0){\circle*{20}}
\put(330,1892){\makebox(320,0)[c]{通信工程}}
\put(1150,1892){\makebox(0,0){通信工程(多媒体通信)}}
% \put(704,1805){\framebox{\parbox[t][25mm][t]{85mm}{\setlength{\baselineskip}{10mm} \center {三维场景描述语言解析器的设计与实现}}}}
\put(704,1765){%
\begin{minipage}[t][35mm][t]{85mm}
\setlength{\baselineskip}{10mm}
\centering {三维场景描述语言解析器的设计与实现}
\end{minipage}
}
\put(704,1715){\makebox(848,200){}}
\put(1087,1453){\makebox(0,0){王友鹏}}
\put(1087,1339){\makebox(0,0){媒体通信081}}
\put(1087,1232){\makebox(0,0){2012.2.20--2012.6.8}}
%\put(1087,1120){\makebox(0,0){专业实验室}}
%\put(1087,1008){\makebox(0,0){王少东}}
\put(1087,1120){\makebox(0,0){王少东}}
\put(1029,571){\makebox(0,0){2012}}
\put(1227,571){\makebox(0,0){02}}
\put(1348,571){\makebox(0,0){18}}
\end{overpic}
\begin{overpic}{eps/assignment_blank_2.eps}
% \put(0,0){\circle*{20}}
\put(297,2157){\parbox[b][27mm][t]{162mm}{
\setlength{\baselineskip}{9mm} 

BART 项目所提供的四个场景数据:\par
http://www.ce.chalmers.se/research/group/graphics/BART/
}}
\put(297,1441){\parbox[b][56mm][t]{130mm}{
\setlength{\baselineskip}{9mm} 
\CTEXindent

在三维场景的描述和定义方面,存在多种不同的格式与方法。
本课题以科研和工程实际中的一种三维场景描述语言为对象,研究对于此类语言的解析方法。\par
要求编写相应的程序,实现基本的解析器功能,作为对于解析方法有效性的验证。
要求所设计并实现的解析器,有较好的性能和可扩展性。
}}
\put(297,922){\parbox[b][45mm][t]{162mm}{
\setlength{\baselineskip}{9mm} 

毕业设计开题报告,3000字以上。\par
外文参考资料的译文,2000字以上。\par
毕业设计说明书,15000字以上。\par
设计成果及其数字化存储。
}}
\put(297,400){\parbox[b][44mm][t]{142mm}{
\setlength{\baselineskip}{9mm} 

[1]  张德丰, 周灵. VRML虚拟现实应用技术[B]. 电子工业出版社, 2010.\par
[2]  孙宏伟, 王健, 杨百龙, 张树生. 产品三维模型STEP数据到VRML格式的转换技术[J]. 西北工业大学学报, 2010(3). \par
[3]  朱长水, 邵建龙. 基于虚拟现实建模语言实现三维动态场景技术[J]. 电脑知识与技术, 2007(1)
}}
\end{overpic}
% \normalsize
\large
\begin{overpic}{eps/assignment_blank_3.eps}
% \put(0,0){\circle*{20}}
\put(315,737){\parbox[b][157mm][t]{139mm}{
\renewcommand{\arraystretch}{1.3}
\begin{tabular}{p{28mm}p{80mm}p{30mm}}

	第一周至第三周	&
	根据任务书要求,搜集、查阅相关资料,完成开题报告;完成外文文献翻译工作
	&
 \\
	第四周至第五周	&
	深入研究文献,准备工作环境\par
阅读场景描述数据的实例,设计数据结构。\par
选择合适的编程工具,编写短小的试验性程序
	&
 \\
	第六周至第七周	&
	设计整个程序的框架结构\par
分别编写各个功能模块\par
进行可扩展性设计
	&
 \\
	第八周至第九周	&
	完成程序的初始版本\par
设计性能测试方法
	&
 \\
	第十周至第十二周	&
	改进程序,优化程序性能\par
设计并实现测试方案,对结果进行测试\par
形成测试报告
	&
 \\
	第十三周至第十四周	&
	撰写毕业设计说明书\par
做好答辩准备
	&
 \\
	第十五周	&
	毕业设计说明书修订与完善\par
论文评阅
	&
 \\
	第十六周	&
	论文答辩\par
毕业设计材料归档
	&
 \\	&
        	&
 \\	&
        	&
 \\
\end{tabular}
}}
\end{overpic}
\Large
\begin{overpic}{eps/assignment_blank_1.eps}
% \put(0,0){\circle*{20}}
\put(330,1892){\makebox(320,0)[c]{通信工程}}
\put(1150,1892){\makebox(0,0){通信工程(多媒体通信)}}
% \put(704,1805){\framebox{\parbox[t][25mm][t]{85mm}{\setlength{\baselineskip}{10mm} \center {一种应用域描述语言的构造与解释}}}}
\put(704,1765){%
\begin{minipage}[t][35mm][t]{85mm}
\setlength{\baselineskip}{10mm}
\centering {一种应用域描述语言的构造与解释}
\end{minipage}
}
\put(704,1715){\makebox(848,200){}}
\put(1087,1453){\makebox(0,0){周红杰}}
\put(1087,1339){\makebox(0,0){媒体通信081}}
\put(1087,1232){\makebox(0,0){2012.2.20--2012.6.8}}
%\put(1087,1120){\makebox(0,0){专业实验室}}
%\put(1087,1008){\makebox(0,0){王少东}}
\put(1087,1120){\makebox(0,0){王少东}}
\put(1029,571){\makebox(0,0){2012}}
\put(1227,571){\makebox(0,0){02}}
\put(1348,571){\makebox(0,0){18}}
\end{overpic}
\begin{overpic}{eps/assignment_blank_2.eps}
% \put(0,0){\circle*{20}}
\put(297,2157){\parbox[b][27mm][t]{162mm}{
\setlength{\baselineskip}{9mm} 

试卷排版项目案例\par
特定应用领域描述语言需求
}}
\put(297,1441){\parbox[b][56mm][t]{130mm}{
\setlength{\baselineskip}{9mm} 
\CTEXindent

通用程序设计预研难以满足特定应用场合的需求。希望设计一种面向特定问题的语言
DSL(Domain Specific Language),来处理特定任务。\par
要求以考试试卷这类具有特殊格式的教学文档的设计为例,
设计一种面向特定应用语言,设计并实现一个该语言的解释器。
}}
\put(297,922){\parbox[b][45mm][t]{162mm}{
\setlength{\baselineskip}{9mm} 

毕业设计开题报告,3000字以上。\par
外文参考资料的译文,2000字以上。\par
毕业设计说明书,15000字以上。\par
设计成果及其数字化存储。
}}
\put(297,400){\parbox[b][44mm][t]{142mm}{
\setlength{\baselineskip}{9mm} 

[1]  Eric S. Raymond. Making Pictures With GNU PIC[EB/OL]. http://floppsie.comp.glam.ac.uk/Glamorgan/gaius/web/pic.html \par
[2]  胡圣明, 李青山, 陈平. 基于对象的特定域语言构造方法[J]. 西安电子科技大学学报, 2006(1). \par
[3]  周艳明. 基于领域专用语言的应用软件自动生成[J]. 计算机工程与应用, 2003(10).
}}
\end{overpic}
% \normalsize
\large
\begin{overpic}{eps/assignment_blank_3.eps}
% \put(0,0){\circle*{20}}
\put(315,737){\parbox[b][157mm][t]{139mm}{
\renewcommand{\arraystretch}{1.3}
\begin{tabular}{p{28mm}p{80mm}p{30mm}}

	第一周至第三周	&
	根据任务书要求,搜集、查阅相关资料,完成开题报告;完成外文文献翻译工作
	&
 \\
	第四周至第五周	&
	深入研究文献,准备工作环境\par
描述DSL的需求\par
研究不同的实现方案和路径,确定实现方式
	&
 \\
	第六周至第七周	&
	研究手工方式生成目标文档的各种步骤,发现其中的局限\par
设计一个微型的语言系统(词法、语法),尝试它的解释器设计\par
使用DSL设计目标文档
	&
 \\
	第八周至第九周	&
	丰富词法和语法,演变成完整的系统\par
建立测试用例\par
设计测试方法和测试过程
	&
 \\
	第十周至第十二周	&
	完善解释器程序,优化DSL\par
健壮性改进与测试\par
形成测试报告
	&
 \\
	第十三周至第十四周	&
	撰写毕业设计说明书\par
做好答辩准备
	&
 \\
	第十五周	&
	毕业设计说明书修订与完善\par
论文评阅
	&
 \\
	第十六周	&
	论文答辩\par
毕业设计材料归档
	&
 \\	&
        	&
 \\	&
        	&
 \\
\end{tabular}
}}
\end{overpic}
\Large
\begin{overpic}{eps/assignment_blank_1.eps}
% \put(0,0){\circle*{20}}
\put(330,1892){\makebox(320,0)[c]{通信工程}}
\put(1150,1892){\makebox(0,0){通信工程(多媒体通信)}}
% \put(704,1805){\framebox{\parbox[t][25mm][t]{85mm}{\setlength{\baselineskip}{10mm} \center 智能手机上的计步器功能实现}}}
\put(704,1765){%
\begin{minipage}[t][35mm][t]{85mm}
\setlength{\baselineskip}{10mm}
\centering 智能手机上的计步器功能实现
\end{minipage}
}
\put(704,1715){\makebox(848,200){}}
\put(1087,1453){\makebox(0,0){左付磊}}
\put(1087,1339){\makebox(0,0){媒体通信081}}
\put(1087,1232){\makebox(0,0){2012.2.20--2012.6.8}}
%\put(1087,1120){\makebox(0,0){专业实验室}}
%\put(1087,1008){\makebox(0,0){王少东}}
\put(1087,1120){\makebox(0,0){王少东}}
\put(1029,571){\makebox(0,0){2012}}
\put(1227,571){\makebox(0,0){02}}
\put(1348,571){\makebox(0,0){18}}
\end{overpic}
\begin{overpic}{eps/assignment_blank_2.eps}
% \put(0,0){\circle*{20}}
\put(297,2157){\parbox[b][27mm][t]{162mm}{
\setlength{\baselineskip}{9mm} 

}}
\put(297,1441){\parbox[b][56mm][t]{130mm}{
\setlength{\baselineskip}{9mm} 
\CTEXindent

}}
\put(297,922){\parbox[b][45mm][t]{162mm}{
\setlength{\baselineskip}{9mm} 

}}
\put(297,400){\parbox[b][44mm][t]{142mm}{
\setlength{\baselineskip}{9mm} 

}}
\end{overpic}
% \normalsize
\large
\begin{overpic}{eps/assignment_blank_3.eps}
% \put(0,0){\circle*{20}}
\put(315,737){\parbox[b][157mm][t]{139mm}{
\renewcommand{\arraystretch}{1.3}
\begin{tabular}{p{28mm}p{80mm}p{30mm}}
	&
        	&
 \\	&
        	&
 \\	&
        	&
 \\	&
        	&
 \\	&
        	&
 \\	&
        	&
 \\	&
        	&
 \\	&
        	&
 \\	&
        	&
 \\	&
        	&
 \\
\end{tabular}
}}
\end{overpic}
\Large
\begin{overpic}{eps/assignment_blank_1.eps}
% \put(0,0){\circle*{20}}
\put(330,1892){\makebox(320,0)[c]{通信工程}}
\put(1150,1892){\makebox(0,0){通信工程(计算机通信)}}
% \put(704,1805){\framebox{\parbox[t][25mm][t]{85mm}{\setlength{\baselineskip}{10mm} \center 面向人员与任务的网络应用系统设计}}}
\put(704,1765){%
\begin{minipage}[t][35mm][t]{85mm}
\setlength{\baselineskip}{10mm}
\centering 面向人员与任务的网络应用系统设计
\end{minipage}
}
\put(704,1715){\makebox(848,200){}}
\put(1087,1453){\makebox(0,0){潘安明}}
\put(1087,1339){\makebox(0,0){算通081}}
\put(1087,1232){\makebox(0,0){2012.2.20--2012.6.8}}
%\put(1087,1120){\makebox(0,0){专业实验室}}
%\put(1087,1008){\makebox(0,0){王少东}}
\put(1087,1120){\makebox(0,0){王少东}}
\put(1029,571){\makebox(0,0){2012}}
\put(1227,571){\makebox(0,0){02}}
\put(1348,571){\makebox(0,0){18}}
\end{overpic}
\begin{overpic}{eps/assignment_blank_2.eps}
% \put(0,0){\circle*{20}}
\put(297,2157){\parbox[b][27mm][t]{162mm}{
\setlength{\baselineskip}{9mm} 

Open source content management platform: Drupal\par
目标网站需求说明
}}
\put(297,1441){\parbox[b][56mm][t]{130mm}{
\setlength{\baselineskip}{9mm} 
\CTEXindent

以“实习管理系统”为例,分析面向人员与任务的管理系统的框架结构。
着重研究其中元素构成,基础设施,组织方式,
设计恰当的接口形式等。\par
选择合适的实现方式,设计并实现该目标系统。
要求系统满足用户的基本需要。并具备一定的可扩展性。
}}
\put(297,922){\parbox[b][45mm][t]{162mm}{
\setlength{\baselineskip}{9mm} 

毕业设计开题报告,3000字以上。\par
外文参考资料的译文,2000字以上。\par
毕业设计说明书,15000字以上。\par
研究和设计成果及其数字化存储。
}}
\put(297,400){\parbox[b][44mm][t]{142mm}{
\setlength{\baselineskip}{9mm} 

[1]  马建玲. 开放源代码的内容管理系统Drupal[J]. 现代情报, 2007.2. \par
[2]  Drupal Community. Draupal Community Documentation[EB/OL].  http://http://drupal.org/documentation\par
[3]  Mat T. Wilson. Using Drupal[J]. Journal of Web Librarianship. 2009(3). 
}}
\end{overpic}
% \normalsize
\large
\begin{overpic}{eps/assignment_blank_3.eps}
% \put(0,0){\circle*{20}}
\put(315,737){\parbox[b][157mm][t]{139mm}{
\renewcommand{\arraystretch}{1.3}
\begin{tabular}{p{28mm}p{80mm}p{30mm}}

	第一周至第三周	&
	根据任务书要求,搜集、查阅相关资料,完成开题报告;完成外文文献翻译工作
	&
 \\
	第四周至第五周	&
	深入研究文献,准备工作环境\par
撰写目标网站建设需求文档\par
选用合适的平台,创建简单的试验性系统
	&
 \\
	第六周至第七周	&
	研究系统各部分及其组成、接口关系\par
设计试验性目标系统的总体结构\par
分别编写各个功能模块\par
考察目标系统的可扩展性
	&
 \\
	第八周至第九周	&
	撰写目标系统使用说明\par
完成目标系统初始版本\par
确定功能测试方法
	&
 \\
	第十周至第十二周	&
	改进程序,优化程序性能\par
考虑性能测试方法,设计并实现测试方案,对结果进行测试\par
形成测试报告
	&
 \\
	第十三周至第十四周	&
	撰写毕业设计说明书\par
做好答辩准备
	&
 \\
	第十五周	&
	毕业设计说明书修订与完善\par
论文评阅
	&
 \\
	第十六周	&
	论文答辩\par
毕业设计材料归档
	&
 \\	&
        	&
 \\	&
        	&
 \\
\end{tabular}
}}
\end{overpic}
\Large
\begin{overpic}{eps/assignment_blank_1.eps}
% \put(0,0){\circle*{20}}
\put(330,1892){\makebox(320,0)[c]{通信工程}}
\put(1150,1892){\makebox(0,0){通信工程(计算机通信)}}
% \put(704,1805){\framebox{\parbox[t][25mm][t]{85mm}{\setlength{\baselineskip}{10mm} \center 小型企业销售业务管理系统设计}}}
\put(704,1765){%
\begin{minipage}[t][35mm][t]{85mm}
\setlength{\baselineskip}{10mm}
\centering 小型企业销售业务管理系统设计
\end{minipage}
}
\put(704,1715){\makebox(848,200){}}
\put(1087,1453){\makebox(0,0){缪莲莲}}
\put(1087,1339){\makebox(0,0){算通081}}
\put(1087,1232){\makebox(0,0){2012.2.20--2012.6.8}}
%\put(1087,1120){\makebox(0,0){专业实验室}}
%\put(1087,1008){\makebox(0,0){王少东}}
\put(1087,1120){\makebox(0,0){王少东}}
\put(1029,571){\makebox(0,0){2012}}
\put(1227,571){\makebox(0,0){02}}
\put(1348,571){\makebox(0,0){18}}
\end{overpic}
\begin{overpic}{eps/assignment_blank_2.eps}
% \put(0,0){\circle*{20}}
\put(297,2157){\parbox[b][27mm][t]{162mm}{
\setlength{\baselineskip}{9mm} 

}}
\put(297,1441){\parbox[b][56mm][t]{130mm}{
\setlength{\baselineskip}{9mm} 
\CTEXindent

}}
\put(297,922){\parbox[b][45mm][t]{162mm}{
\setlength{\baselineskip}{9mm} 

}}
\put(297,400){\parbox[b][44mm][t]{142mm}{
\setlength{\baselineskip}{9mm} 

}}
\end{overpic}
% \normalsize
\large
\begin{overpic}{eps/assignment_blank_3.eps}
% \put(0,0){\circle*{20}}
\put(315,737){\parbox[b][157mm][t]{139mm}{
\renewcommand{\arraystretch}{1.3}
\begin{tabular}{p{28mm}p{80mm}p{30mm}}
	&
        	&
 \\	&
        	&
 \\	&
        	&
 \\	&
        	&
 \\	&
        	&
 \\	&
        	&
 \\	&
        	&
 \\	&
        	&
 \\	&
        	&
 \\	&
        	&
 \\
\end{tabular}
}}
\end{overpic}
\Large
\begin{overpic}{eps/assignment_blank_1.eps}
% \put(0,0){\circle*{20}}
\put(330,1892){\makebox(320,0)[c]{通信工程}}
\put(1150,1892){\makebox(0,0){通信工程(计算机通信)}}
% \put(704,1805){\framebox{\parbox[t][25mm][t]{85mm}{\setlength{\baselineskip}{10mm} \center {每日自动联编与发布系统的设计与实现}}}}
\put(704,1765){%
\begin{minipage}[t][35mm][t]{85mm}
\setlength{\baselineskip}{10mm}
\centering {每日自动联编与发布系统的设计与实现}
\end{minipage}
}
\put(704,1715){\makebox(848,200){}}
\put(1087,1453){\makebox(0,0){李宁}}
\put(1087,1339){\makebox(0,0){算通081}}
\put(1087,1232){\makebox(0,0){2012.2.20--2012.6.8}}
%\put(1087,1120){\makebox(0,0){专业实验室}}
%\put(1087,1008){\makebox(0,0){王少东}}
\put(1087,1120){\makebox(0,0){王少东}}
\put(1029,571){\makebox(0,0){2012}}
\put(1227,571){\makebox(0,0){02}}
\put(1348,571){\makebox(0,0){18}}
\end{overpic}
\begin{overpic}{eps/assignment_blank_2.eps}
% \put(0,0){\circle*{20}}
\put(297,2157){\parbox[b][27mm][t]{162mm}{
\setlength{\baselineskip}{9mm} 

实验用软件项目git仓库\par
项目部署的目标环境
}}
\put(297,1441){\parbox[b][56mm][t]{130mm}{
\setlength{\baselineskip}{9mm} 
\CTEXindent

研究软件项目演进试开发的工作流程,
根据任务书所提供的软件项目案例,
设计一种自动化的工作方案,实现项目
自动化的Nightly Build。\par
在实际工作时,源代码管理系统以git为例,
操作系统,主要考虑Linux环境。
}}
\put(297,922){\parbox[b][45mm][t]{162mm}{
\setlength{\baselineskip}{9mm} 

毕业设计开题报告,3000字以上。\par
外文参考资料的译文,2000字以上。\par
毕业设计说明书,15000字以上。\par
设计成果及其数字化存储。
}}
\put(297,400){\parbox[b][44mm][t]{142mm}{
\setlength{\baselineskip}{9mm} 

[1]  杨锦方. CVS和Nightly Build技术[B]. 清华大学出版社, 2002.10. \par
[2]  Joel Spolsky. JOEL说软件[B]. 电子工业出版社, 2005.9. \par
[3]  苟振兴. 把握项目的脉搏[J]. 程序员, 2004(06).
}}
\end{overpic}
% \normalsize
\large
\begin{overpic}{eps/assignment_blank_3.eps}
% \put(0,0){\circle*{20}}
\put(315,737){\parbox[b][157mm][t]{139mm}{
\renewcommand{\arraystretch}{1.3}
\begin{tabular}{p{28mm}p{80mm}p{30mm}}

	第一周至第三周	&
	根据任务书要求,搜集、查阅相关资料,完成开题报告;完成外文文献翻译工作
	&
 \\
	第四周至第五周	&
	深入研究文献,准备工作环境\par
研究源代码管理工具、自动化编译工具工作原理和运行机制\par
采用手工方式编译、部署项目,研究其中的环节
	&
 \\
	第六周至第七周	&
	设计自动编译、发布与部署的自动化工作流程\par
使用脚本语言编写项目管理脚本\par
在不同的环境下,考察脚本的兼容性
	&
 \\
	第八周至第九周	&
	完成脚本程序\par
设计测试方法和测试用例
	&
 \\
	第十周至第十二周	&
	改进脚本程序,优化脚本程序性能\par
设计并实现测试方案,对结果进行测试\par
形成测试报告
	&
 \\
	第十三周至第十四周	&
	撰写毕业设计说明书\par
做好答辩准备
	&
 \\
	第十五周	&
	毕业设计说明书修订与完善\par
论文评阅
	&
 \\
	第十六周	&
	论文答辩\par
毕业设计材料归档
	&
 \\	&
        	&
 \\	&
        	&
 \\
\end{tabular}
}}
\end{overpic}
\Large
\begin{overpic}{eps/assignment_blank_1.eps}
% \put(0,0){\circle*{20}}
\put(330,1892){\makebox(320,0)[c]{通信工程}}
\put(1150,1892){\makebox(0,0){通信工程(计算机通信)}}
% \put(704,1805){\framebox{\parbox[t][25mm][t]{85mm}{\setlength{\baselineskip}{10mm} \center {一种CMS的结构分析与扩展研究}}}}
\put(704,1765){%
\begin{minipage}[t][35mm][t]{85mm}
\setlength{\baselineskip}{10mm}
\centering {一种CMS的结构分析与扩展研究}
\end{minipage}
}
\put(704,1715){\makebox(848,200){}}
\put(1087,1453){\makebox(0,0){徐明华}}
\put(1087,1339){\makebox(0,0){算通081}}
\put(1087,1232){\makebox(0,0){2012.2.20--2012.6.8}}
%\put(1087,1120){\makebox(0,0){专业实验室}}
%\put(1087,1008){\makebox(0,0){王少东}}
\put(1087,1120){\makebox(0,0){王少东}}
\put(1029,571){\makebox(0,0){2012}}
\put(1227,571){\makebox(0,0){02}}
\put(1348,571){\makebox(0,0){18}}
\end{overpic}
\begin{overpic}{eps/assignment_blank_2.eps}
% \put(0,0){\circle*{20}}
\put(297,2157){\parbox[b][27mm][t]{162mm}{
\setlength{\baselineskip}{9mm} 

Open source content management platform: Drupal\par
目标网站需求说明
}}
\put(297,1441){\parbox[b][56mm][t]{130mm}{
\setlength{\baselineskip}{9mm} 
\CTEXindent

以Drupal为例,分析内容管理系统(CMS)的框架结构。
着重研究Drupal的元素构成,基础设施,组织方式,
接口形式等。\par
为了检验研究成果的正确性和有效性,设定一个由实际案例
中抽象出来的系统为目标,通过设计和实现该目标系统,
来改进Drupal系统的应用水平。
}}
\put(297,922){\parbox[b][45mm][t]{162mm}{
\setlength{\baselineskip}{9mm} 

毕业设计开题报告,3000字以上。\par
外文参考资料的译文,2000字以上。\par
毕业设计说明书,15000字以上。\par
研究和设计成果及其数字化存储。
}}
\put(297,400){\parbox[b][44mm][t]{142mm}{
\setlength{\baselineskip}{9mm} 

[1]  马建玲. 开放源代码的内容管理系统Drupal[J]. 现代情报, 2007.2. \par
[2]  张平杉. 利用开源CMS建立基于Web2.0的图书馆门户网站的实践[J]. 企业导报, 2011(17). \par
[3]  Drupal Community. Draupal Community Documentation[EB/OL].  http://http://drupal.org/documentation\par
[4]  Mat T. Wilson. Using Drupal[J]. Journal of Web Librarianship. 2009(3). 
}}
\end{overpic}
% \normalsize
\large
\begin{overpic}{eps/assignment_blank_3.eps}
% \put(0,0){\circle*{20}}
\put(315,737){\parbox[b][157mm][t]{139mm}{
\renewcommand{\arraystretch}{1.3}
\begin{tabular}{p{28mm}p{80mm}p{30mm}}

	第一周至第三周	&
	根据任务书要求,搜集、查阅相关资料,完成开题报告;完成外文文献翻译工作
	&
 \\
	第四周至第五周	&
	深入研究文献,准备工作环境\par
撰写目标网站建设需求文档\par
以Drupal为平台,创建简单的试验性系统
	&
 \\
	第六周至第七周	&
	研究Drupal的各部分及其组成、接口关系\par
设计试验性目标系统的总体结构\par
分别编写各个功能模块\par
考察目标系统的可扩展性
	&
 \\
	第八周至第九周	&
	撰写对于Drupal平台框架研究的成果文档\par
完成目标系统初始版本\par
确定功能测试方法
	&
 \\
	第十周至第十二周	&
	改进程序,优化程序性能\par
考虑性能测试方法,设计并实现测试方案,对结果进行测试\par
形成测试报告
	&
 \\
	第十三周至第十四周	&
	撰写毕业设计说明书\par
做好答辩准备
	&
 \\
	第十五周	&
	毕业设计说明书修订与完善\par
论文评阅
	&
 \\
	第十六周	&
	论文答辩\par
毕业设计材料归档
	&
 \\	&
        	&
 \\	&
        	&
 \\
\end{tabular}
}}
\end{overpic}
\end{center}
\end{document}
